\documentclass[11pt]{article}

\usepackage{comment}
\usepackage{wrapfig}
\usepackage{url}
\usepackage{color}
\usepackage{marvosym}
\usepackage{enumerate}
\usepackage{subfig}
\usepackage{amsmath}
\usepackage{amssymb}
\usepackage{hyperref}
\usepackage{tikz}
\usepackage{pgfplots}

\usepackage{caption}
\usepackage{graphicx}

\oddsidemargin 0mm
\evensidemargin 5mm
\topmargin -20mm
\textheight 240mm
\textwidth 160mm


\setlength{\parskip}{.5cm plus4mm minus3mm}

\def\ci{\perp\!\!\!\perp}

\newcommand{\vw}{{\bf w}}
\newcommand{\vx}{{\bf x}}
\newcommand{\vy}{{\bf y}}
\newcommand{\vxi}{{\bf x}_i}
\newcommand{\yi}{y_i}
\newcommand{\vxj}{{\bf x}_j}
\newcommand{\vxn}{{\bf x}_n}
\newcommand{\yj}{y_j}
\newcommand{\ai}{\alpha_i}
\newcommand{\aj}{\alpha_j}
\newcommand{\X}{{\bf X}}
\newcommand{\Y}{{\bf Y}}
\newcommand{\vz}{{\bf z}}
\newcommand{\msigma}{{\bf \Sigma}}
\newcommand{\vmu}{{\bf \mu}}
\newcommand{\vmuk}{{\bf \mu}_k}
\newcommand{\msigmak}{{\bf \Sigma}_k}
\newcommand{\vmuj}{{\bf \mu}_j}
\newcommand{\msigmaj}{{\bf \Sigma}_j}
\newcommand{\pij}{\pi_j}
\newcommand{\pik}{\pi_k}
\newcommand{\D}{\mathcal{D}}
\newcommand{\el}{\mathcal{L}}
\newcommand{\N}{\mathcal{N}}
\newcommand{\vxij}{{\bf x}_{ij}}
\newcommand{\vt}{{\bf t}}
\newcommand{\yh}{\hat{y}}
\newcommand{\code}[1]{{\footnotesize \tt #1}}
\newcommand{\alphai}{\alpha_i}
%\newcommand{\ci}{\perp\!\!\!\perp}

\newif\iftfanswer
\newif\ifunanswered
\newcommand{\answertruefalse}[1][\unansweredtrue]{
\unansweredfalse
#1
\ifunanswered
    \underline{Unanswered}
\else
    \iftfanswer
        \textbf{True}
    \else
        \textbf{False}
\fi
}
\newcommand{\answertruefalseN}[1][-1]{
\ifnum#1<0
    \underline{Unanswered}
\else
    \ifnum#1=0
        \textbf{False}
    \else
        \textbf{True}
    \fi
\fi}
\newcommand{\tftrue}{\answertruefalseN[1]}
\newcommand{\tffalse}{\answertruefalseN[0]}
\newcommand{\tfunanswered}{\answertruefalseN[-1]}
\newcommand{\inputsolution}[1]{\underline{\bf Solution}:

\input{#1}
}
% \newcommand{\odv}[2]{\frac{d #1}{d#2}}
\newcommand{\odv}[3][1]{
    \ifnum #1 =1
        \frac{d #2}{d#3}
    \else
        \frac{d^{#1} #2}{d#3^{#1}}
    \fi
}

\newcommand{\pdv}[3][1]{
    \ifnum #1 = 1
        \frac{\partial #2}{\partial#3}
    \else
        \frac{\partial^{#1} #2}{\partial#3^{#1}}
    \fi
}
\newcommand{\indep}{\mathrel{\perp\mkern-10mu\perp}} % independence
\newcommand{\given}{~\vert~}
\newcommand{\condindep}[3]{(#1 \indep #2 \given #3)} % conditional independence
\newcommand{\expect}[1]{\mathbb{E}\left[#1\right]}
\newcommand{\likelihood}{\mathcal{L}}
\DeclareMathOperator{\relu}{ReLU}
\DeclareMathOperator{\argmax}{argmax}
\DeclareMathOperator{\argmin}{argmin}
\DeclareMathOperator{\sign}{sign}

\newcounter{questionnumber}
\stepcounter{questionnumber}

\newcommand{\questionnumber}{\noindent \arabic{questionnumber}\stepcounter{questionnumber})~~}
\newcommand{\truefalse}[1]{\questionnumber #1\\True~~~~~~~~False\\Explanation:\\\\ }%\vspace{3cm}}

\pagestyle{myheadings}
\markboth{}{Fall 2021 CS 475-675 Machine Learning: Midterm 1}


\title{CS 475-675 Machine Learning: Midterm 1\\
\Large{Fall 2021}\\
150 points. }
\author{}
\date{}


\begin{document}
\large
\maketitle
\thispagestyle{headings}

\vspace{-.5in}
\noindent Name (print):
\underline{\makebox[5in][l]{Dimitri Lezcano}} \\

\noindent JHED:
\underline{\makebox[5in][l]{dlezcan1}}


 \vspace{3cm}
 If you think a question is unclear or multiple answers are reasonable, please write a brief explanation of your answer,
 to be safe. Also, show your work if you want wrong answers to have a chance at some credit: it lets us see how much you understood.

 This exam is open-book: permitted materials include textbooks, personal notes, lecture material, recitation material, past assignments, the course Piazza, and scholarly articles and papers. Other materials are otherwise not permitted. It is not permitted to discuss or share questions or solutions of this exam with any person, via any form of communication, other than the course instructors.  It is not permitted to solicit or use any solutions to past exams for this course.
 \vspace{1cm}

 \textbf{ Declaration:}

I have neither given nor received any unauthorized aid on this exam. In particular, I have not spoken to any other student about any part of this exam.
The work contained herein is wholly my own.  I understand that violation of these rules, including using an unauthorized aid, copying from another person,
or discussing this exam with another person in any way, may result in my receiving a 0 on this exam.
\begin{center}
\noindent\underline{\makebox[6in][l]{}}

 Signature ~~~~~~~~~~~~~~~~~~~~~~~~~~~~~~~~~~~~~~~~~~~~~~~~~Date


 \vspace{3cm}
 Good luck!
 \end{center}


\newpage

%\section*{Glossary}
%
%\begin{itemize}
%
%\item 
%
%\end{itemize}
%
%\newpage

\section*{True/False (50 points)}

For each question, circle (or otherwise clearly indicate) either True or False. If False, explain why.\\
2 points for correct True/False answer, -2 points for incorrect True/False answer, 3 points for a correct explanation, 0 points for an incorrect explanation.\\

\truefalse{
Minimizing the exponential loss is a good way of creating a classifier robust to outliers.
}
\begin{align*}
    J(\beta) &= (\mathbf{y}-\mathbf{X}\beta)^T(\mathbf{y}-\mathbf{X}\beta)+\lambda\sum_i\beta_i^2 \\
    &= (\mathbf{y}^T-\beta^T\mathbf{X}^T)(\mathbf{y}-\mathbf{X}\beta)+\lambda\sum_i\beta_i^2 \\
    &= \mathbf{y}^T\mathbf{y}-\mathbf{y}^T\mathbf{X}\beta - \beta^T\mathbf{X}^T\mathbf{y}^T+\beta^T\mathbf{X}^T\mathbf{X}\beta+\lambda\sum_i\beta_i^2 \\
    &= \mathbf{y}^T\mathbf{y}-2\mathbf{y}^T\mathbf{X}\beta+\beta^T\mathbf{X}^T\mathbf{X}\beta+\lambda\sum_i\beta_i^2
\end{align*}
To minimize $J(\beta)$, we set $\frac{\partial J(\beta)}{\partial \beta} = 0$
\begin{align*}
    \frac{\partial J(\beta)}{\partial \beta} &=-2\mathbf{X}^T\mathbf{y} + 2\mathbf{X}^T\mathbf{X}\beta + 2\lambda \beta = 0 \\
    \mathbf{X}^T\mathbf{X}\beta + \lambda\mathbf{I}\beta &= \mathbf{X}^T\mathbf{y} \\
    (\mathbf{X}^T\mathbf{X} + \lambda\mathbf{I})\beta &= \mathbf{X}^T\mathbf{y} \\
    \beta &= (\mathbf{X}^T\mathbf{X} + \lambda\mathbf{I})^{-1}\mathbf{X}^T\mathbf{y}
\end{align*}

\truefalse{Minimising the 0-1 loss on the training data can never achieve good out of sample prediction performance.}
\begin{align*}
p(p|f\cap not\; c)&=\frac{p(f\cap not\; c|p)\times p(p)}{p(f\cap not\; c)} \\
&=\frac{p(f|p)\times p(not \; c|p)\times p(p)}{p(f)-p(f \cap c)} \\
&=\frac{p(f|p)\times (1- p(c|p))\times p(p)}{p(f)-p(f \cap c)} \\
&=\frac{(0.5)\times (1-0.7)\times (0.05)}{(0.2-0.08)}=\frac{0.0075}{0.12}=0.0625
\end{align*}

\truefalse{Naive Bayes is a linear classifier.}
\begin{align*}
    p(\lambda |x) &\propto p(x|\lambda)p(\lambda) \\
    &\propto \lambda^ne^{-\lambda \sum_i x_i} \lambda^{\alpha -1}e^{-\beta \lambda} \\
    &\propto e^{-\lambda(\sum_ix_i-\beta)}\lambda^{n+\alpha-1} \\
    p(\lambda |x) &\propto Gamma(\alpha^*, \beta^*), where \; \alpha^* = \alpha +n \; and \; \beta^* =\sum_ix_i+\beta
\end{align*}



\newpage

\truefalse{Decreasing the bias of a predictor must necessarily increase its variance.}
% \newcommand{\odv}[2]{\frac{d #1}{d#2}}
\newcommand{\odv}[3][1]{
    \ifnum #1 =1
        \frac{d #2}{d#3}
    \else
        \frac{d^{#1} #2}{d#3^{#1}}
    \fi
}

\newcommand{\pdv}[3][1]{
    \ifnum #1 = 1
        \frac{\partial #2}{\partial#3}
    \else
        \frac{\partial^{#1} #2}{\partial#3^{#1}}
    \fi
}
\begin{enumerate}[(i) ]
    \item $a_1 = -1 + 6 - 2 = 3$, $h_1 = 3$, $\yh = h_1 + 2 x_1 = 3 - 2 = 1$
    
    
    \item $\hat{y} = ReLU(w_1 x_1 + w_2 x_2 + w_3 x_3) + 2 x_1$
    
    
    \item The gradients of $l$ w.r.t $w_i$ is given by 
    $$
    \pdv{l}{w_i} = \pdv{l}{\yh}\pdv{\yh}{w_i} = -(y - \yh)\pdv{\yh}{w_i} = 
    \begin{cases}
    -(y-\yh) x_i, & w_1 x_1 + w_2 x_2 + w_3 x_3 > 0\\
    0, &\text{otherwise}
    \end{cases}
    $$
    We find that $\yh = ReLU(-1 + 6 + 2) - 2 = 5$. So $(y - \yh) = -1 - 5 = 6$.
    Therefore
    $$
    \left.\pdv{l}{w_1} = -6 (-1) = 6 ~\right|~
    \left.\pdv{l}{w_2} = -6 (-3) = 18 ~\right|~
    \pdv{l}{w_3} = -6 (-2) = 12
    $$
    
    
    \item In each case, we find that the update will follow the negative of the gradient direction and the update here $\Delta w_i \propto -\pdv{l}{w_i} < 0 ~\forall i = 1,2,3$, so each parameter will decrease.
    
    
    \item $\pdv{l}{v_1} = \pdv{l}{\yh}\pdv{\yh}{v_1} = -(y-\yh)x_1 = -6(-1) = 6$ and $\pdv{l}{v_2} = \pdv{l}{\yh}\pdv{\yh}{v_2} = -(y-\yh)h_1 = -6(7) = -42$. $v_{i,new} = v_i - \eta \pdv{l}{v_i}$, so we have $v_{1,new} = 2 - \frac{1}{2}(6) = -1$ and $v_{2,new} = 1 - \frac{1}{2}(-42) = 22$.
    
    
\end{enumerate}

\truefalse{$\mathbb{E}[\mathbb{E}[A \mid B]] = \mathbb{E}[A]$.}
\newcommand{\Graph}{\mathcal{G}}
\tftrue 

Suppose that $\Graph_1 = (V, E_1)$ and $\Graph_2 = (V, E_2)$ are different undirected graphs. Then (WLOG) $\exists$ at least a single pair $v_1, v_2 \in V$ s.t. $(v_1, v_2) \in E_1 \cap E_2'$. Therefore, since the graphs are MRFs, we can write the distributions of each of these graphs as
$$
p_{\Graph_i}(\vec{v}) = \frac{1}{Z} \prod_{v \in V} \phi(v) \prod_{(v_j, v_k) \in E_i} \phi(v_j, v_k).
$$

Note that $p_{\Graph_1} \neq p_{\Graph_2} \iff \exists p_{\Graph_1}(\vec{v})/p_{\Graph_1}(\vec{v}) \neq 1 $. So consider $p_{\Graph_1}(\vec{v})/p_{\Graph_2}(\vec{v})$
\begin{align*}
    \frac{p_{\Graph_1}(\vec{v})}{p_{\Graph_2}(\vec{v})} &= \frac{\frac{1}{Z} \prod_{v \in V} \phi(v) \prod_{(v_j, v_k) \in E_1} \phi(v_j, v_k)}{\frac{1}{Z} \prod_{v \in V} \phi(v) \prod_{(v_j, v_k) \in E_2} \phi(v_j, v_k)}\\
        &= \frac{\prod_{(v_j, v_k) \in E_1} \phi(v_j, v_k)}{\prod_{(v_j, v_k) \in E_2} \phi(v_j, v_k)}\\
        &= \frac{\prod_{(v_j, v_k) \in E_1\cap E_2'} \phi(v_j, v_k)}{\prod_{(v_j, v_k) \in E_2\cap E_1'} \phi(v_j, v_k)} ~\text{which is at least}\\
        &= \phi(v_1, v_2) \frac{\prod_{(v_j, v_k) \in E_1\cap E_2'\setminus(v_1, v_2)} \phi(v_j, v_k)}{\prod_{(v_j, v_k) \in E_2\cap E_1'\setminus(v_1, v_2)} \phi(v_j, v_k)} \neq 1 \because \phi(v_1, v_2) \neq 1 ~\text{generally}
\end{align*}
Therefore, two different MRFs results in distributions that are not equivalent. Furthermore, we have that in $\Graph_1$, ($v_1 \indep $ non-neighbors $\given v_2$ and other neighbors), but in $\Graph_2$, ($v_1 \indep$ non-neighbors $\given$ neighbors that don't include $v_2$). This results in a different independence which yields a different distribution.


\truefalse{Rosenblatt's Perceptron algorithm can be modified to solve multiclass classification problems, provided any pair of classes is linearly separable.}
\tftrue

For each class $Y_l$ (say $l = 1, \dots, k$), define a perceptron with parameters $\beta_l$ such that 
$f_l(x) = \sign(\beta_l^T x)$ (including bias term in dot product) be the decision rule for $x$ to be classified as $Y_l$ or not $Y_l$. So here, we have $k$ perceptron classifiers, $f_l$, that determines whether each is either in a specific class or not. Since all of the classes are linearly separable, the class $Y_l$ is linearly separable with not $Y_l$ classes. So we train each perceptron to discriminate what is $Y_l$ and what is not $Y_l$ for $l = 1, \dots, k$, since these classes are linearly separable with what is $Y_l$ and what is not $Y_l$. 

\newpage

\truefalse{Naive Bayes is a special case of a Markov random field.}
\tftrue 

Naive Bayes assumes $X_i \ci X_1, \dots, X_{i-1}, X_{i+1}, \dots, X_k \given Y ~\forall i = 1, \dots, k$. Take a graphical model, $\mathcal{G}$ where $Y$ is the root node and $X_i$ are all children of $Y$. In other words $\mathcal{G} = (\mathcal{V}, \mathcal{E})$ where $\mathcal{V} = \{ X_1, \dots, X_k, Y \}$ are the vertices and $\mathcal{E} = \{ (X_i, Y) \given  i = 1, \dots, k \}$ are the pair-wise edges of $\mathcal{G}$. Therefore, the maximal cliques of this graph are all of size 2 and we have the independences, $X_i \ci X_1, \dots, X_{i-1}, X_{i+1}, \dots, X_k \given Y ~\forall i = 1, \dots, k$, the same independence as Naive Bayes. Therefore by Hammersley-Clifford theorem, Naive Bayes factorizes w.r.t $\mathcal{G} \iff$ Naive Bayes obeys the global and local markov properties $\iff$ Naive Bayes is an MRF. 

% \emph{I think: Go through MRF properties}

\truefalse{If $A {\perp\!\!\!\perp} C$ then either $A {\perp\!\!\!\perp} B$ or $B {\perp\!\!\!\perp} C$.}
\tffalse

Value iteration will always converge to the true value function in an infinite number of steps. We are not guaranteed convergence of a finite number of steps.

\truefalse{A support vector machine applied to a linearly separable dataset must have at least $2$ support vectors.}
\tftrue 

The goal for an SVM is to maximize the margin between two classes. Therefore, in order to calculate a margin, at the minimum, it will need one vector from each class that are the closest to the (linear) decision boundary to define the margin. Otherwise, the margin may not be maximized should the SVM only have one support vector (i.e. maximum between a single support vector is $\infty$).

\newpage

\truefalse{Logistic regression is a special case of the projection pursuit model with $M=1$.}
\tffalse

\underline{Projection pursuit models}:  $M$ different linear combinations of features $\beta_m^T \vec{x}$ to learn a complex non-linear link functions and add the results
$$
f(\vec{x}) = \sum_{m=1}^M g_m(\beta_m^T \vec{x})
$$
for a set of \emph{unrestricted} $g_m(\cdot)$.

Here, logistic regression is actually a restricted moment model which learns $f(\vec{x}) = g(\vec{x}; \beta)$ where $g$ is \emph{fixed}, not unrestricted. Therefore, we have that logisitc regression is NOT a projection pursuit model.

\pagebreak
\section*{Multiple Part Questions (100 points)}

\questionnumber {\bf Probability And Likelihood (25 points)}

\begin{itemize}

\item[(i)] Show that if $A \ci B \cup D \mid C$ then $A \ci D \mid C$.  You may want to use the fact that if $X \ci Y \mid Z$ then $p(X, Y \mid Z) = p(X \mid Z) p(Y \mid Z)$.

\item[(ii)] Consider the following density function (for non-negative $x$): $f(x; \lambda) = \lambda \exp \left\{ - \lambda x \right\}$.  Derive the maximum likelihood estimate of its parameter $\lambda$.

\item[(iii)] What function of $x$ does $\lambda$ correspond to?

\item[(iv)] If, instead of solving for $\lambda$ in closed form, we instead opted to use stochastic gradient descent to update our guess $\lambda_{(t)}$ to obtain $\lambda_{(t+1)}$, using the $i$th row of data $x_i$, and a learning parameter $\rho$, what would our update be?

\end{itemize}

\textbf{Solution:}

\begin{enumerate}[(i)]
    \item 
    \begin{align*}
        p(A, D \given C) &= \sum_b p(A, B=b, D \given C)\\
        &= \sum_b p(A \given C) p(B = b, D \given C) \because A \ci B \cup D \given C\\
        &= p(A \given C) \sum_b p(B = b, D \given C)\\
        &= p(A \given C) p(D \given C) \implies A \ci D \given C
    \end{align*}
    
    \item
    \begin{align*}
        \log\likelihood_{[D]}(\lambda) &= \sum_i \log(f(x_i ; \lambda))\\
            &= \sum_i -\lambda x_i + \log \lambda\\
        \pdv{}{\lambda}\log\likelihood_{[D]} &= \sum_i -x_i + \frac{1}{\lambda} = 0 \implies \\
        -\left(\sum_i x_i\right) + \frac{N}{\lambda} &= 0\\
        \frac{1}{\lambda} &= \frac{1}{N} \sum_i x_i \\
        \lambda &= \left(\frac{1}{N} \sum_i x_i\right)^{-1}
    \end{align*}
    
    \item $\lambda_{MLE}$ corresponds to the inverse of the mean value of $x$. 
    
    \item Using (ii), we would find
    \begin{align*}
        \pdv{}{\lambda_i} \log \likelihood_{(x_i)}(\lambda_i) &= -x_i + \frac{1}{\lambda_i}\\
        \lambda_{i+1} &= \lambda_i - \rho \left(\pdv{}{\lambda_i} \log \likelihood_{(x_i)}(\lambda_i)\right)\\
            &= \lambda_i - \rho\left(-x_i + \frac{1}{\lambda_i}\right)
    \end{align*}
    
    
\end{enumerate}

\newpage

\questionnumber {\bf Classification (25 points)}

Consider the \emph{Slightly Less Naive Bayes} classifier on features $X_1, \ldots, X_k$ and outcome $Y$, defined by the following independence restrictions:
\begin{align*}
X_1 &\ci X_{3}, \ldots, X_{k} \mid X_{2},Y \\
X_k &\ci X_{1}, \ldots, X_{k-2} \mid X_{k-1},Y \\
X_i &\ci X_{1}, \ldots, X_{i-2}, X_{i+2}, \ldots, X_k \mid X_{i-1},X_{i+1},Y
\end{align*}

\begin{itemize}
\item[(i)] Write down the joint likelihood ${\cal L}_{[D]} \equiv \prod_{i=1}^n p(\vec{x}_i, y_i)$ for this model.  You may want to think about the MRF factorization corresponding to the above independences.

\item[(ii)] Write down the predictive model $p(Y \mid \vec{X})$ for this classifier.

\item[(iii)] Write down the decision boundary for classes $Y_j$ and $Y_m$: $\log \frac{p(Y_j \mid \vec{X})}{p(Y_m \mid \vec{X})}$.  Is this a linear decision boundary in $\vec{X}$?

\item[(iv)] Is the Naive Bayes model a submodel of the Slightly Less Naive Bayes model?  Recall that a model is a set of distributions (that satisfy some independence restrictions in this case), and a 
submodel is a subset.  In other words, all distributions in a submodel (subset) are also in a supermodel (superset).  That is, elements of a submodel satisfy restrictions in a supermodel.

Hint: think about the graphoid axioms, and how Naive Bayes independences compare to Slightly Less Naive Bayes independences.

\end{itemize}

\textbf{Solution:}

\begin{enumerate}[(i)]
    \item Note that the distribution $p(\vec{X} \given Y)$ can be modelled as an MRF chain for $$X_1 \given Y \text{---} X_2 \given Y \text{---} \dots \text{---} X_{k-1} \given Y \text{---} X_{k} \given Y $$ 
    Now, considering a single sample $(\vec{x}_i, y_i)$ and using the MRF factorization of maximal cliques for a 
    \begin{align*}
        p(\vec{x_i}, y_i) &= p(y_i) p(\vec{x_i} \given y_i) \\
            &= p(y_i) \frac{1}{Z} \prod_{j=1}^k \phi_{j | y_i}(x_i^{(j)} \given y_i) \prod_{j=1}^{k-1} \phi_{j,j+1 | y_i}(x_i^{(j)}, x_i^{(j+1)} \given y_i)\\
        \likelihood_{[D]} &= \prod_{i=1}^n p(\vec{x_i}, y_i)\\
            &= \prod_{i=1}^n \frac{1}{Z} p(y_i) \prod_{j=1}^k \phi_{j | y_i}(x_i^{(j)} \given y_i) \prod_{j=1}^{k-1} \phi_{j,j+1 | y_i}(x_i^{(j)}, x_i^{(j+1)} \given y_i)
    \end{align*}
    
    \item Using Bayes' rule,
    \begin{align*}
        p(Y \given \vec{X}) &= \frac{p(\vec{X} \given Y) p(Y)}{\sum_y p(\vec{X} \given Y = y)} \\
            &= p(Y) \frac{\frac{1}{Z}\prod_{j=1}^k \phi_{j | Y}(X_j \given Y) \prod_{j=1}^{k-1} \phi_{j,j+1 | Y}(X_j, X_{j+1} \given Y)}{\sum_y \frac{1}{Z} \prod_{j=1}^k \phi_{j | y}(X_j \given Y = y) \prod_{j=1}^{k-1} \phi_{j,j+1 | y}(X_j, X_{j+1} \given Y = y)}\\
        p(Y \given X) 
            &= p(Y) \frac{\prod_{j=1}^k \phi_{j | Y}(X_j \given Y)         \prod_{j=1}^{k-1} \phi_{j,j+1 | Y}(X_j, X_{j+1} \given Y)}{\sum_y \prod_{j=1}^k \phi_{j | y}(X_j \given Y = y) \prod_{j=1}^{k-1} \phi_{j,j+1 | y}(X_j, X_{j+1} \given Y = y)}
    \end{align*}
    
    \item Using (ii) and changing $Y_j \rightarrow Y_l$,
    \begin{align*}
        \log \frac{p(Y_l \given \vec{X})}{p(Y_m \given \vec{X})} &= \log\left( \frac{p(Y_l)}{p(Y_m)}\frac{\prod_{j=1}^k \phi_{j | Y_l}(X_j \given Y_l)         \prod_{j=1}^{k-1} \phi_{j,j+1 | Y_l}(X_j, X_{j+1} \given Y_l)}{\prod_{j=1}^k \phi_{j | Y_m}(X_j \given Y_m)         \prod_{j=1}^{k-1} \phi_{j,j+1 | Y_m}(X_j, X_{j+1} \given Y_m)} \right)\\
            &= \log\left(\frac{p(Y_l)}{p(Y_m)} \prod_{j=1}^k \frac{\phi_{j | Y_l}(X_j \given Y_l)}{\phi_{j | Y_m}(X_j \given Y_m)} \prod_{j=1}^{k-1} \frac{\phi_{j,j+1 | Y_l}(X_j, X_{j+1} \given Y_l)}{\phi_{j,j+1 | Y_m}(X_j, X_{j+1} \given Y_m)}\right)\\
            &= \log{\frac{p(Y_l)}{p(Y_m)}} + \sum_{j=1}^k \log{\frac{\phi_{j | Y_l}(X_j \given Y_l)}{\phi_{j | Y_m}(X_j \given Y_m)}} + \sum_{j=1}^{k-1}\log{\frac{\phi_{j,j+1 | Y_l}(X_j, X_{j+1} \given Y_l)}{\phi_{j,j+1 | Y_m}(X_j, X_{j+1} \given Y_m)}}\\
            &= \log{\frac{p(Y_l)}{p(Y_m)}} + \sum_{j=1}^k\left(\log\phi_{j | Y_l}(X_j \given Y_l) - \phi_{j | Y_m}(X_j \given Y_m)\right) \sum_{j=1}^{k-1}(\log\phi_{j,j+1 | Y_l}(X_j, X_{j+1} \given Y_l) \\
            &~- \phi_{j,j+1 | Y_m}(X_j, X_{j+1} \given Y_m))
    \end{align*}
    The above decision boundary is linear in $\vec{X}$ if $\forall j =1, \dots, k, \forall Y$ we have that $\log\phi_{j | Y}$ and $\log\phi_{j,j+1 | Y}$ is linear in $\vec{X}$.
    
    \item Naive Bayes' assumes
    \begin{align*}
    X_1 &\ci X_2, \dots, X_k \given Y \\
    X_k &\ci X_1, \dots, X_{k-2} \given Y\\
    X_i &\ci X_1, \dots, X_{i-1}, X_{i+1}, \dots, X_k \given Y ~\forall i = 2, \dots, k-1
    \end{align*}
    Given that we know using the graphoid axiom, \emph{weak union}, $A \ci B, C \given D \implies A \ci B \given C, D$, we have that
    \begin{align*}
    X_1 &\ci X_2, \dots, X_k \given Y \implies X_i \ci X_3, \dots, X_k \given X_2, Y\\  
    X_k &\ci X_1, \dots, X_{k-2} \given Y \implies X_k \ci X_1, \dots, X_{k-3} \given X_{k-2}, Y\\
    X_i &\ci X_1, \dots, X_{i-1}, X_{i+1}, \dots, X_k \given Y  \implies X_i \ci X_1, \dots, X_{i-2}, X_{i+2}, \dots, X_k \given X_{i-1}, X_{i+1}, Y\\ &~\forall i = 2, \dots, k-1
    \end{align*}
    So Naive Bayes is a submodel of Slightly Less Naive Bayes. 
    
\end{enumerate}

\newpage



\questionnumber {\bf Neural Network Models (25 points)}

Consider a neural network with $K$ inputs $\vec{X}$, and output $Y$.


\begin{itemize}

\item[(i)] Assume there are $M>10$ hidden layers with one node each.  The first hidden node $V_1$ uses the sigmoid activation function $\sigma(x) = 1/(1 + e^{-x})$ applied to a linear combination of $\vec{X}$ given by $\vec{\beta}_0^T \vec{X}$, all subsequent hidden nodes $V_m$ ($m = 2, \ldots, M$) use a sigmoid activation function applied to a weighted combination of the value of the previous hidden node $V_{m-1}$ and a bias term ($V_m = \text{sigmoid}(\beta_{m0} + \beta_{m1} V_{m-1})$), with the output $Y$ being given by $\text{sigmoid}(\beta_{M0} + \beta_{M1} V_M)$.

Describe how the gradients with respect to $\vec{\beta}_0$ differ between layers. Justify your answer.
Hint: calculate the derivative of $\sigma(x)$, and think about what will happen by chain rule of differentiation if a set of $\sigma(x)$ functions are composed.

\item[(ii)] Now replace all $\sigma(x)$ activation functions in the model in (i) by ReLU activation functions. 

How does the gradient of the output with respect to $\vec{\beta}_0$ compare with that same gradient in (i)?

\item[(iii)] Assume there is a single hidden layer with $K$ nodes, each with a $ReLU(x)$ activation function, and each feature in $\vec{X}$ is connected to exactly one hidden node.  In other words, each hidden node $V_k$ ($k = 1, \ldots, K$) is equal to $ReLU(\beta_k X_k)$.  Furthermore, assume the output $Y$ is given by $\sigma(\vec{\beta}^T (V_1, \ldots, V_K))$.  Is this model equivalent to the logistic regression model?  Explain.

\item[(iv)] As before, assume there is a single hidden layer with $K$ nodes, each with a $\sigma(x)$ activation function.  However, now each hidden node $V_k$ ($k = 1, \ldots, K$) is given by
$\sigma()$ applied to a weighted combination of $\vec{X}$, where weights for each $V_k$ are \emph{shared}.  The outcome model is given by a weighted combination of $\vec{\alpha}^T (V_1, \ldots, V_K)$.  Is this model sufficiently general to be able to approximate the logistic regression model?  Explain.

\end{itemize}

\textbf{Solution:}

\begin{enumerate}[(i)]
    \item First, note that $\odv{\sigma}{x} = \odv{}{x} (1 + e^{-x})^{-1} = -e^{-x}(1 + e^{-x})^{-2} = -\sigma(x)(1 - \sigma(x))$. 
    
    Second, using the above, note that $$\pdv{V_m}{V_{m-1}} = \pdv{}{V_{m-1}} \sigma(\beta_{m0} + \beta_{m1} V_{m-1}) = \beta_{m1} V_m ( 1 - V_m)$$
    
    Therefore if we consider just looking at the bias terms,
    \begin{align*}
        \pdv{V_M}{\beta_{m0}} &= \pdv{V_M}{V_{M-1}}\pdv{V_{M-1}}{\beta_{m0}} 
             = \pdv{V_m}{\beta_{m0}}\prod_{j=m+1}^M \pdv{V_j}{V_{j-1}}\\
            &= V_m( 1 - V_m) \prod_{j=m+1}^M \beta_{j1} V_j ( 1 - V_j )
    \end{align*}
    Now since $V_j$ is a sigmoid, $ 0 \leq V_j(1 - V_j) \leq \frac{1}{2} < 1$, therefore, as $m$ gets smaller (closer to $1$, closer to input layer) we see that the product term becomes smaller $\because$
    \begin{equation*}
        \prod_{j=m+1}^M V_j ( 1 - V_j) \leq \prod_{j=m+1}^M \frac{1}{2} \leq \left(\frac{1}{2}\right)^{M - m - 1}
    \end{equation*}
    Similarly, 
    \begin{align*}
        \pdv{V_M}{\beta_{m1}} &= \pdv{V_m}{\beta_{m1}} \prod_{j=m+1}^M \pdv{V_j}{V_{j-1}}
    \end{align*}
    So we would have that as $m \rightarrow 1$, the proportionality of the gradients decays to zero ($\prod_{j=m+1}^M \pdv{V_j}{V_{j-1}}$). This is also known as the \emph{vanishing gradient} problem.
    
    \item First, note that $$\odv{\relu(x)}{x} = \begin{cases} 1 & x > 0\\ 0 & x \leq 0 \end{cases}$$
    
    Second, note that
    $$
    \pdv{V_j}{V_{j-1}} = \pdv{}{V_{j-1}} \relu(\beta_{j0} + \beta_{j1} V_{j-1}) = \mathbb{I}_{>0}(\beta_{j0} + \beta_{j1}V_{j-1}) \beta_{j1}
    $$
    Therefore if we consider just looking at the bias terms,
    \begin{align*}
        \pdv{V_M}{\beta_{m0}} &= \pdv{V_M}{V_{M-1}}\pdv{V_{M-1}}{\beta_{m0}} 
             = \pdv{V_m}{\beta_{m0}}\prod_{j=m+1}^M \pdv{V_j}{V_{j-1}}\\
             &= \mathbb{I}_{> 0}(\beta_{m0} + \beta_{m1}V_{m-1})\prod_{j=m+1}^M  \mathbb{I}_{>0}(\beta_{j0} + \beta_{j1}V_{j-1}) \beta_{j1}
    \end{align*}
    where for parameters that need to be updated, the indicator function for each of these higher layers would be $1$, which would mean that the update is $\propto \prod_{j=m+1}^M \beta_{j1}$, rather than any decay terms. Similarly, just as in (i), the gradients for $\beta_{m1}$ would have the same proportionality constant of $1$.
    
    \item We have here that $Y = \sigma(\beta^T \relu(X))$ while logistic regression $Y_{\log} = \sigma(\beta^T X)$, which does not have a non-linear activation ($\relu$). In more mathematical terms
    \begin{align*}
        \sigma(\beta^T \relu(X)) &= \sigma(\beta^T X)~\forall X  ~\text{iff}\\
        \exp(-\beta^T \relu(X)) &= \exp(-\beta^T X)~\forall X  ~\text{iff}\\
        \beta^T \relu(X) &= \beta^T X ~\forall X ~\text{iff}\\
        % \beta^T(\relu(X) - X) &= 0 ~\forall X ~\text{iff}\\
        \relu(X) &= X ~\forall X ~\text{which is a contradiction}
    \end{align*}
    So no, this is not a logistic regression model.
    
    \item The model's output is represented by $Y = \sum_{k=1}^K \alpha_k \sigma(\beta^T X)$. If we set $\alpha_1 = 1$ and $\alpha_{k > 1} = 0$, we have that $ Y = \sigma(\beta^T X) = Y_{\log}$, so the model is sufficiently general approximate logistic regression.
    
    
\end{enumerate}

\newpage



\questionnumber {\bf AdaBoost and Exponential Loss (25 points)}

%\begin{itemize}
%
%\item[(i)]
%
%\end{itemize}

Assume a one-dimensional dataset with $x = \langle -1, 0, 1 \rangle$ and $y =
\langle -1, +1, -1 \rangle$.  Consider three weak classifiers:
\begin{align*}
  h_1(x) = \begin{cases}
    1  & \text{if } x > \frac{1}{2} \\
    -1 & \text{otherwise}
  \end{cases},
  \quad\quad
  h_2(x) = \begin{cases}
    1  & \text{if } x > - \frac{1}{2} \\
    -1 & \text{otherwise}
  \end{cases},
  \quad\quad
  h_3(x) = 1.
\end{align*}

\begin{enumerate}[(1)]
  %\newsubquestion

\item Show your work for the first $2$ iterations of \textsc{AdaBoost}.
  In the table below, replace the ``?'' in the table below with the value of the
  quantity at iteration $t$.
  For the $\varepsilon_t$ column, put its error rate (remember to use the correct row weights).  
  For the $h$ column, put the index of the weak learner with the best error rate at the current iteration.
  For the $\alpha_t$ column, put the update parameter appropriately calculated from $\varepsilon_t$.
  In addition, calculate the distribution over examples at the second iteration $D_{2}(1), D_{2}(2), D_{2}(3)$.
  Use the natural logarithm in your calculations ($\log$ with base $e$).

\begin{tabular}{c|ccc|ccc}
$t$ & $h_t$ & $\alpha_t$ & $\varepsilon_t$ & $D_{t}(1)$ & $D_{t}(2)$ & $D_{t}(3)$ \\ \hline
1 & ? & ? & $ ? $ & $ \frac{1}{3} $ & $ \frac{1}{3} $ & $ \frac{1}{3} $ \\
2 & ? & ? & $ ? $ & $ ? $ & $ ? $ & $ ? $ \\
\end{tabular}

\item Without performing an explicit calculation, make an argument for what the classifier will be that AdaBoost outputs in this case after 100 iterations?  Explain.

\item In general, getting near-perfect training accuracy in machine learning leads to overfitting. However, \textsc{AdaBoost} can get perfect training accuracy without overfitting. Give a brief justification for why that is the case.
  
\item The logistic regression likelihood (with outcome labels $y \in \{ 1, -1 \}$) is equal to $\prod_{i=1}^{n} \frac{1}{1 + \exp(- y_i \vec{\beta} \vec{x}_i)}$.  Show that maximizing the likelihood is equivalent to minimizing the exponential loss: $\sum_{i=1}^n \exp(- y_i f(\vec{x}_i))$ for some $f(.)$.

\end{enumerate}

\textbf{Solution:}

\begin{enumerate}[(1)]
    \item 
    \begin{tabular}{c|ccc|ccc}
    $t$ & $h_t$ & $\varepsilon_t$ & $\alpha_t$ & $D_{t}(1)$ & $D_{t}(2)$ & $D_{t}(3)$ \\ \hline
    1 & 2 & $1/3$ & 0.3466 & $ 1/3 $ & $ 1/3 $ & $ 1/3 $ \\
    2 & 2 & $1/3$ & 0.3466 & $ 1/4 $ & $ 1/4 $ & $ 1/2 $ \\
    3 & 2 & $1/3$ & 0.3466 & $ 1/6 $ & $ 1/6 $ & $ 2/3 $
    \end{tabular}
    
    \item What we can see here is that $f_{Adaboost}(x) = h_2(x)$ will end up being the classifier after 100 iterations. What we can see here is that $h_2$ correctly classifies $x = -1 \& x = 0$ and misclassifies $x = +1$. $h_1 \& h_3$ both misclassify $x = +1$ as well as misclassify one of either $x = 0$ or $x = -1$. Therefore, $h_2$ will always be picked as the best weak learner therefore since its loss will always out-perform $h_1$ and $h_3$ since it's error rate will always be less as $x = +1$ will continued to get weighted heavier, but all of the classifiers misclassify $x = +1$. Therefore, no other learners will be selected by Adaboost, so $h_2(x)$ will be the learned classifier.
    
    \item AdaBoost minimizes the exponential loss of the classifier which focuses very heavily on misclassified data, increasing generalization error. 
    
    \item 
    \begin{align*}
        \argmax_\beta \prod_i \frac{1}{1 + \exp(-y_i \beta x_i)} &= \argmin_\beta \prod_i (1 + \exp(-y_i \beta x_i)) \\
            &= \argmin \log \prod_i (1 + \exp(-y_i \beta x_i))\\
            &= \argmin \sum_i \log(1 + \exp(-y_i \beta x_i))\\
            &\approx \argmin \sum_i \exp(-y_i \beta x_i) ~\text{using a first-order approximation.}
    \end{align*}
    Therefore we see that for $f(x) = \beta x$, this is equivalent as minimizing exponential loss (for logistic regression fits that are within a first-order approximation).
    
    
\end{enumerate}


\end{document}

